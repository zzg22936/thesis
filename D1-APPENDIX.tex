\newpage
\chapter*{附录一\quad 主要缩写符号对照表}
\vskip 5mm

\begin{tabular}{p{0.15\columnwidth}p{0.85\columnwidth}}
	ED  &  欧式距离(Euclidean Distance)  \\
	DTW  & 动态时间卷曲(Dynamic Time Warping)  \\
	LCSS  & 最长公共子串(Longest Common Subsequence)  \\
	EDR  & 实值序列上的编辑距离(Edit Distance on Real sequence)  \\
	ERP  & 带有惩罚项的编辑距离(Edit Distance with  Real Penalty)  \\
	HD & 霍斯托夫距离(Hausdorff Distance)  \\
	FD  & 弗雷歇距离(Fréchet Distance)  \\
	OWD  & 一路距离 (One Way Distancen)  \\
	DEwP  & 带投影的编辑距离 (Edit Distance with   Projections)  \\
	P2P  & 点对点 (Peer-To-Peer) \\
	DP  & 道格拉斯-普克 (Douglas-Peuker)  \\
	SVD & 奇异值分解 (Singular Value Decomposition)  \\
	DFT  & 离散傅里叶变换 (Discrete Fourier Transform) \\
	HWT & 哈尔小波变换 (Haaar Wavelet Transform)  \\
	PAA & 分段聚合近似(Piecewise Aggregate Approximation)  \\
	MDL & 最小描述长度(Minimal Description Length)\\
	APCA & 自适应分段常量近似(Adaptive Piecewise Constant Approximation)\\
	SED & 欧式距离的平方 (Squared Euclidean Distance)\\
	BE & 包围信封 (Bounding Envelope)\\
	MBR & 最小包围矩形 (Minimum Bounding Rectengle)\\
	BF& 界特征 (Bound Feature)\\
\end{tabular}

\clearpage
\phantom{s}
\clearpage

\chapter*{附录二\quad 基于LCSS距离的查询实现}
\vskip 5mm
最长公共子串距离的计算公式如公式\ref{eq:LCSS}所示。
\begin{equation}\label{eq:LCSS}
	LCSS({\cal Q},{\cal C})= func(n, m)
\end{equation}
\begin{equation*}	
func(i,j)=
\begin{cases}
0  & if \quad i=0 \lor j=0,\\ 
1 + func(i-1, j-1) & if \quad d(q_{i}, c_{j}) \le \delta, \\
max 
\begin{cases}
func(i-1,j),\\
func(i,j-1),
\end{cases} & otherwise
\end{cases}
\end{equation*}

我们对$\cal Q$使用章节\ref{sec-c5-BE}所用包围盒技术进行多粒度概要数据计算,并以所得到的多粒度包围信封作为概要数据。此时,记第$l$层包围盒为${\cal Q}_{l}= \{qL_{l},qU_{l}\}$,其第$i$个元素$\vq_{l,i}=\{qU_{l,i}, qL_{l,i} \} $。此时针对分别来自$\cal Q$和$\cal C$的两个点$\vq_{i}$ 和 $\vc_{j}$,假设我们不知道$\vq_{i}$的真实值,但可以通过包围盒知道改点属于$\vq_{l,i}$所表示的范围内。那么我们可以计算出$\vq_{i}$ 和 $\vc_{j}$距离的下界$d_{lb}$和上界$d_{ub}$。
\begin{lemma}\label{lemma:SEDLUB}
	$d_{lb}(\bq_{l,i},\bc_{j})\le d(\bq_{i},\bc_{j})  \le d_{ub}(\bq_{l,i},\bc_{j})$, 其中 $\bq_{l,i}=\langle qL_{l,i}, qU_{l,i} \rangle$, $d_{lb}(\bq_{l,i},\bc_{j})$ 和 $d_{ub}(\bq_{l,i},\bc_{j})$ 的计算方式如下:
\allowdisplaybreaks[4]
\end{lemma}
\allowdisplaybreaks
 	\begin{equation}
 d_{lb}(\vq_{l,i},\vc_{j})=
 \sum_{k=1}^{d} \begin{cases}
 ({\vc}_{j}^{k}- {qU}_{l,i}^{k})^2 & \emph{if} \,\,\,  {\vc}_{j}^{k} > {qU}_{l,i}^{k},\\
 \quad 0 &   \emph{if} \,\,\,   {qL}_{l,i}^{k} \le \vc_{j}^{k} \le {qU}_{l,i}^{k},\\
 ({qL}_{l,i}^{k} -{\vc_{j}}^{k})^2 & \emph{if} \,\,\,    {\vc}_{j}^{k} <{qL}_{l,i}^{k}.\\
 \end{cases}
 \end{equation}
 \allowdisplaybreaks[4]
 \allowdisplaybreaks
 \begin{equation}
 d_{ub}(\vq_{l,i},\vc_{j})=
 \sum_{k=1}^{d} \begin{cases}
 ({\vc}_{j}^{k}- {qL}_{l,i}^{k})^2 & \emph{if} \,\,\,  {\vc}_{j}^{k} > {qU}_{l,i}^{k},\\
 ({qU}_{l,i}^{k}- {qL}_{l,i}^{k})^2 &   \emph{if} \,\,\,   {qL}_{l,i}^{k} \le \vc_{j}^{k} \le {qU}_{l,i}^{k},\\
 ({qU}_{l,i}^{k} -{\vc_{j}}^{k})^2 & \emph{if} \,\,\,    {\vc}_{j}^{k} < {qL}_{l,i}^{k}.\\
 \end{cases}
 \end{equation}
 上述引理可根据欧式距离的定义直观看出。接下来,我们将展示$d_{lb}$和上界$d_{ub}$会随着细粒度的概要数据的使用,而逐渐逼近真实值。
 \begin{lemma} \label{lemma:tighterPointBounds}
 	$d_{ub}(\bq_{l+1,i}, \bc_{j}) \le d_{ub}(\bq_{l,i}, \bc_{j})$ , $d_{lb}(\bq_{l+1,i}, \bc_{j}) \ge d_{lb}(\bq_{l,i}, \bc_{j})$.
 \end{lemma}
 \begin{proof}
 	既然 $qL_{l,i}^{k} \le qL_{l+1,i}^{k}  \le \vq_{i}^{k} \le qU_{l+1,i}^{k} \le qU_{l,i}^{k}$,  
 	$k \in{[1,d]} $,  那么我们得到如下分析:
 	\begin{equation}	
 	\begin{cases}
 	(\vc_{j}^{k} - qL_{l+1,i}^{k} )^{2} \le (\vc_{j}^{k} - qL_{l,i}^{k})^{2} & \emph{if} \,\,\,   \vc_{j}^{k} >qU_{l,i}^{k}\\ 
 	(\vc_{j}^{k} - qL_{l+1,i}^{k} )^{2} \le   (qU_{l,i}^{k} - qL_{l,i}^{k})^{2}                  & \emph{if} \,\,\,   qU_{l+1,i}^{k} < \vc_{j}^{k} \le qU_{l,i}^{k} ,\\ 
 	\quad 0 = 0                    & \emph{if} \,\,\,   qL_{l+1,i}^{k} < \vc_{j}^{k} \le qU_{l+1,i}^{k} ,\\ 
 	(qU_{l+1,i}^{k}- \vc_{j}^{k})^{2} \le (qU_{l,i}^{k}- qL_{l,i}^{k})^{2}                     &\emph{if} \,\,\,  qL_{l,k}^{j} \le \vc_{j}^{k} \le qL_{l+1,i}^{k} ,\\ 
 	(qU_{l+1,i}^{k}- \vc_{j}^{k})^{2} \le 	( qU_{l,i}^{k}- \vc_{j}^{k})^{2}   & \emph{otherwise}
 	\end{cases}
 	\end{equation}
 	将所有维度数据相加 $d_{ub}(\vq_{l+1,i}, \vc_{j}) \le d_{ub}(\vq_{l,i}, \vc_{j})$。对于下界,同理可以进行分析
 	%	 we can proof  $d_{lb}(\bq_{l+1}^{i}, \bc_{j}) \ge d_{lb}(\bq_{l}^{i}, \bc_{j})$.
 	\begin{equation}	
 	\begin{cases}
 	(\vc_{j}^{k} - qU_{l+1,i}^{k} )^{2} \ge (\vc_{j}^{k} - qU_{l,i}^{k})^{2} & \emph{if} \,\,\,    \vc_{j}^{k} >qU_{l,i}^{k}\\ 
 	(\vc_{j}^{k} - qU_{l+1,i}^{k} )^{2} \ge 0  & \emph{if} \,\,\,   qU_{l+1,i}^{k} < \vc_{j}^{k} \le qU_{l,i}^{k} ,\\ 
 	\quad	0 = 0                    & \emph{if} \,\,\,   qL_{l+1,i}^{k} < \vc_{j}^{k} \le qU_{l+1,i}^{k} ,\\ 
 	( qL_{l+1,i}^{k}- \vc_{j}^{k})^{2} \ge 0                    &\emph{if} \,\,\,   qL_{l,k}^{j} \le \vc_{j}^{k} \le qL_{l+1,i}^{k} ,\\ 
 	( qL_{l+1,i}^{k}- \vc_{j}^{k})^{2} \ge 	( qL_{l,i}^{k}- \vc_{j}^{k})^{2}   & \emph{otherwise}
 	\end{cases}
 	\end{equation}
 	将所有维度数据累加,我们得到  $d_{lb}(\vq_{l+1,i}, \vc_{j}) \ge d_{lb}(\vq_{l,i}, \vc_{j})$ 成立。 
 \end{proof}

最后,我们介绍针对LCSS距离设计基于多粒度包围信封的上、下界。其中上界记为 $UB\_LCSS({\cal Q}_{l}, {\cal C})$,其计算方法如公式\ref{eq:ubLCSS}所示。下界记为$LB\_LCSS({\cal Q}_{l}, {\cal C})$,其计算方式为将$UB\_LCSS$定义中的$ d_{lb}$用$ d_{ub}$替代。
\begin{equation}\label{eq:ubLCSS}
UB\_LCSS({\cal Q}_{l},{\cal C})= func'(n, m)
\end{equation}
\begin{equation*}	
func'(i,j)=
\begin{cases}
0  & if \quad i=0 \lor j=0,\\ 
1 + func'(i-1, j-1) & if  d_{lb}(\vq_{l,i}, \vc_{j}) \le \delta, \\
max 
\begin{cases}
func'(i-1,j)\\
func'(i,j-1)
\end{cases} & otherwise
\end{cases}
\end{equation*}
	\begin{theorem}\label{theory:LCSSupper}$UB\_LCSS({\cal Q}_{l}, {\cal C}) \ge 	LCSS({\cal Q},{\cal C})$,即
	$UB\_LCSS({\cal Q}_{l}, {\cal C})$	是 $LCSS({\cal Q},{\cal C})$的上界。
\end{theorem}  
\begin{proof}\label{proof:LCSSUPPER}
	假定$Z=\{ z_{1}, z_{2}, \cdots, z_{k}\}$是$LCSS({\cal Q},{\cal C})$ 计算所求得的最长公共子串。那么对于任意属于$Z$的元素$z_{d}$,一定存在着一对元素$(\vq_{i}, \vc_{j})$ 
	使得$d(q_{i}, c_{j}) \le \delta$ 。根据$d_{lb}$定义,我们有 $d_{lb}(q_{l,i}, c_{j}) \le d(q_{i}, c_{j}) \le \delta$。 所以 $Z$ 也是$UB\_LCSS({\cal Q}_{l},{\cal C})$ 的一个公共子串,且它的长度小于或等于$UB\_LCSS({\cal Q}_{l}, {\cal C})$所蕴含的最长公共子串。所以,$UB\_LCSS({\cal Q}_{l}, {\cal C})\ge LCSS({\cal Q},{\cal C})$。
\end{proof}

	\begin{theorem}\label{theory:LCSSlower}$LB\_LCSS({\cal Q}_{l}, {\cal C}) \le 	LCSS({\cal Q},{\cal C})$,即
	$LB\_LCSS({\cal Q}_{l}, {\cal C})$	是 $LCSS({\cal Q},{\cal C})$的下界。
\end{theorem}  
\begin{proof}
证明同定理\ref{theory:LCSSupper}。
\end{proof}
进一步地,我们将展示所提上、下界会随着包围信封粒度的增加而逐渐变紧。
\begin{theorem}\label{theory:LCSSupperDecrease}
	$UB\_LCSS({\cal Q}_{l+1}, {\cal C}) \le UB\_LCSS({\cal Q}_{l}, {\cal C}) $。
\end{theorem}  

\begin{proof}\label{proof:LCSSDecrease}
假设 $Z=\{ z_{1}, z_{2}, \cdots, z_{k}\}$ 是$UB\_LCSS ({\cal Q}_{l+1}, {\cal C})$所蕴含的最长公共子串。
那么对于$Z$中任意一点$z_{d}$ 必然存在着一对元素 $(\vq_{i}, \vc_{j})$ 满足 $d_{lb}(\vq_{l+1,i}, \vc_{j}) \le \delta$。根据$d_{lb}$定义,我们有 $d_{lb}(q_{l,i}, c_{j}) \le d_{lb}(q_{l+1,i}, c_{j})$。因此,$Z$ 也是 $UB\_LCSS({\cal Q}_{l}, {\cal C})$的一条公共子串, 且$Z$的长度小于其最长公共子串的长度。因此我们有
$UB\_LCSS({\cal Q}_{l+1}, {\cal C})\le UB\_LCSS({\cal Q}_{l}, {\cal C})$。 
\end{proof}
\begin{theorem}\label{theory:LCSSlowerDecrease}
$LB\_LCSS({\cal Q}_{l+1}, {\cal C}) \ge LB\_LCSS({\cal Q}_{l}, {\cal C}) $。
\end{theorem}  
\begin{proof}
证明同定理\ref{theory:LCSSupperDecrease}。
\end{proof}

至此,我们已经为LCSS距离设计了基于包围信封的概要数据,以及不断逼紧的距离上、下界。因而,满足了使用FTB框架 的条件。可以基于FTB框架实现针对LCSS距离的查询。
 \clearpage
 \phantom{s}
 \clearpage

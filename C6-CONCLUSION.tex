\chapter{总结与展望}

本文重点研究了基于矩阵分解相关理论的个性化推荐系统。本章对全文研究内容和技术模型进行总结归纳,并展望未来的重点研究工作。

\section{研究总结}


互联网发展以来,大量用户和企业产生了海量数据,面临着信息过载的问题。用户需要在大量无用信息中寻找少量自身感兴趣的信息,而企业也需要寻找与企业产品相关的目标用户,进行产品精准营销和推广。针对上述问题,个性化推荐系统应运而生,能够根据用户访问商品的历史行为数据建模,刻画用户和商品之间的交互关系,将可能感兴趣的商品推荐给目标用户。在使用互联网服务时,用户在网上寻找目标商品,遇到满意商品进行消费,然后进行显式评分,并利用标签描述商品的特点,且使用文本评价表达自己使用商品的感受。本文重点研究了上述的推荐场景,旨在解决三个相关问题:(1)\textit{商品推荐};(2)\textit{评分预测};(3)\textit{标签推荐}。\textit{商品推荐}主要是根据用户访问商品的历史行为,根据用户的个人喜好,推荐用户感兴趣的商品。\textit{评分预测}则是根据用户历史评分行为,预测用户对特定商品的满意程度。\textit{标签推荐}则是利用用户的历史使用标签的信息和特定商品的属性信息,个性化地推荐标签给用户来描述商品,方便用户输入。\textit{商品推荐}是推荐系统的主要目标,而好的\textit{评分预测}系统和\textit{标签推荐}系统则是有助于用户体验,并且促进推荐系统的良性循环。具体地,本文的研究问题和技术贡献总结如下。

首先,针对\textit{商品推荐},本文基于隐式反馈数据提出一种加权局部矩阵分解模型。传统加权矩阵分解模型基于全局低秩的假设,不能对隐式反馈数据的局部信息建模。而本文提出的模型设计了高效的子矩阵选择算法建模数据的局部信息,并改进了交替最小二乘算法进行子矩阵加权分解,刻画用户和商品的局部内在特征。并且,该模型带来两个额外好处,缓解了数据稀疏性问题和更好地进行分布式矩阵分解。基于公开的真实数据集的实验验证了该模型的有效性。

其次,针对\textit{评分预测},本文强调了局部矩阵分解模型在显式评分数据上的局部可解释性和目标一致性。近期的工作不能对数据的局部信息进行直观解释,并且分为两个步骤进行局部矩阵分解。基于此,本文结合主题模型和概率矩阵分解,提出多主题矩阵分解模型,该模型确保在同⼀个⽬标函数⾥分别对局部信息和用户、商品局部特征建模。此外,本文还利用狄利克雷分布和高斯-韦斯特分布作为模型参数的先验分布,得到全贝叶斯的多主题矩阵分解模型,使得模型对参数设置不敏感,并且能获得更加准确的评分预测。基于公开的真实数据集的实验验证了该模型在显式评分数据上进行评分预测的有效性,并能够对数据局部信息做出一定的解释。
	
最后,针对\textit{标签推荐},本文强调了标签的时间信息对标签推荐的帮助。现有的基于协同过滤模型的工作没有考虑用户使用标签的时间因素,并且协同过滤模型对新用户冷启动的标签推荐不友好。针对上述问题,本文提出了时间感知的张量分解模型,利用时间点过程对标签时间信息建模,并和逐对排序张量分解模型结合。基于公开数据集的实验表明该模型在准确性上优于当前流行的个性化标签推荐算法,同时具有可接受的推荐新颖性。

\section{研究展望}
本文重点研究了在用户的隐式反馈数据、显式评分数据和显式标签数据上的三个推荐系统相关问题。作者认为还可以从下面三个方面进行扩展。

首先,本文针对隐式反馈数据和显式评分数据进行了局部矩阵分解,需要对局部信息建模,增加了模型训练的时间和空间复杂度,尽管本文提出了一些改进的优化算法,但局部矩阵分解模型时间大于之前的单个全局模型。基于此,未来需要研究更高效的优化算法,加快模型参数学习效率。
	
其次,本文使用了较为规整的用户访问次数数据,评分数据以及标签数据。但推荐数据还包括许多其他数据,类似用户的评论文本数据,描述商品的图片数据以及领域数据(例如签到数据中地点位置信息,音乐中的音频数据等)。一些前人工作~\cite{he2015vbpr,li2016point,wang2017your,zheng2017joint}已经表明这类数据对提高个性化推荐系统的准确率都有极大的帮助。因此,如何融合多源数据也是未来研究的重点方向。
	
最后,本文提出的模型都是基于矩阵分解或者张量分解模型,都属于线性模型。但用户访问、评分、描述商品的行为是非常复杂的,受很多因素影响。因此,使用非线性模型,例如深度学习模型~\cite{wang2015collaborative,he2017neural},来捕捉这类复杂关系,能够更好地刻画用户画像和商品特性,也是未来研究的重点方向。

\clearpage
\phantom{s}
\clearpage
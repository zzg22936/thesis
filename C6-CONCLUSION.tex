\chapter{总结与展望}

本文重点研究了分布式轨迹数据上的$k$近邻查询。本章对全文研究内容和技术模型进行总结归纳,并展望未来的重点研究工作。

\section{研究总结}

随着移动设备的广泛应用,各行各业产生了海量的同时分布在各存储结点的轨迹数据,面临着如何利用该数据的问题。轨迹$k$近邻查询是轨迹数据挖掘研究的重要内容,也是基于位置服务的基本功能。针对海量轨迹数据,分布式场景下的$k$近邻查询问题应运而生。本文重点研究了基于协调者结点-远程结点分布式架构下的$k$近邻查询,旨在从以下三个方面解决相关问题:
(1)\textit{目前存在着许多轨迹距离度量方式,缺乏适用于所有度量方式的查询处理框架}。设计统一的处理框架可以满足不同应用场景的需求,使得处理的问题不再受限。
(2)\textit{如何降低查询的通信开销;}相对于传统集中式环境下的查询,分布式环境下通信开销成为算法的首要瓶颈。
(3)\textit{如何降低查询的时间开销;}提高查询的处理效率,对查询处理有着重要意义。
。具体地,本文的研究问题和技术贡献总结如下。

首先,针对\textit{缺乏适用于所有度量方式的查询处理框架},本文基于多粒度概要数据模型和基于概要数据的上、下界提出了两种查询处理框架:FTB和FLB。 传统的近邻查询针对具体的距离度量方式,设计具体的查询实现算法。而本文提出的框架能满足不同距离度量方式的需求。其中FTB适用于那些根据概要数据能同时计算出上、下界的场景。而FLB适用于那些根据概要数据仅能计算出距离下界的场景。
针对具体的距离度量方式,我们只需研究出适合的概要数据并提供界信息,便能应用到对应的框架中。

其次,针对\textit{分布式查询的通信开销问题},本文强调了通信开销对查询方法性能的影响。现有基于集中式环境和分布式环境的查询往往关注方法的时效性,鲜有考虑通信开销对查询算法的影响。而通信开销是分布式算法的性能的首要指标。基于此本文设计的使用概要数据进行剪枝的思想能大大减少所要传输的数据量。进一步地,我们分别针对欧式距离设计了基于小波变换的多粒度概要数据,
针对动态时间卷曲距离设计了基于包围信封的多粒度概要数据。本文设计的迭代式算法将概要数据由粗到细粒度发送到远程结点,并在每个结点进行剪枝。理论分析和基于公开数据集的实验表明本文所提方法能显著降低查询的通信开销。
	
最后,针对\textit{分布式$k$近邻轨迹查询的时效问题},本文强调了时间效率队查询方法性能的影响。现有的查询方法大都使用索引的方式来提高查询效率。但本文处理的分布式场景中,不要求各个节点构建轨迹索引。为此,本文提出通过构建距离上、下界进行剪枝的策略能有效地降低查询方法的复杂度。特别地,针对欧式距离应用的场景,设计在更新上、下界的过程中进行剪枝的策略以进一步降低计算开销。针对动态时间卷曲距离应用的场景,引入了使用多个下界进行级联剪枝和在更新下界过程中剪枝的两种剪枝策略,以提高查询效率。理论分析和基于公开数据集的实验表明本文所提方法能显著提高查询的时间效率。

\section{研究展望}
本文重点研究了分布式$k$近邻轨迹查询问题。作者认为还可以从下面三个方面进行扩展。

首先,本文针对语义轨迹上的查询可以进行拓展研究。本文所研究的轨迹数据来自欧式空间,如何结合含非欧空间的语义信息进行近邻查询会是一个有趣的问题。一些前人工作\cite{Xiao,Kaiser,WangBCSSQ17}已经考虑了对语义轨迹上的查询。因此,基于轨迹的查询也是未来研究的重点。
	
其次,本文研究了所提两个查询框架分别在欧式和动态时间卷曲这两个常用距离的具体实现。附件部分还指出了在最长公共子串距离下的应用。但轨迹距离度量方式还有许多,在以后的工作中需要针对具体的度量方式设计适合的概要数据,并为概要数据提供距离范围计算方法。

	
最后,本文所提算法是针对协调者-远程结点这一分布式场景设计的查询算法。随着区块链\footnote{https://blockchain.info/}、以太坊\footnote{https://www.ethereum.org/}等基于P2P系统和概念的兴起,基于P2P这一分布式架构的轨迹等时间序列上的近邻查询会成为未来的热点。
\clearpage
\phantom{s}
\clearpage
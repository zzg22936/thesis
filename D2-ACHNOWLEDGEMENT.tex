{\kaishu
\chapter*{致\qquad 谢}
\begin{spacing}{1.5}
	
	
回首十一年的本、硕、博三阶段求学时光,经历了失落与荣耀。求学路上,目睹了一届届同学离开校园走上工作岗位的背影。此刻的我,即将博士毕业,感慨良多。在此期间,有幸得到老师、同学和亲友们的指导与帮助,在此谨对他们表示衷心的感谢!
	
首先,我要郑重感谢我人生中的三位引路人。他们分别是吉根林、周傲英和金澈清三位教授。
吉根林教授是我在南京师范大学本、硕阶段的指导老师。他对待学生视若己出,为我们创造了良好的学习环境与和热烈活跃的团队氛围。并在我即将硕士毕业的十字路口,为我介绍了人生中第二位引路人周傲英教授。
周老师是国内首屈一指的数据库领域专家,他在数据库领域的远瞩高瞻、对新兴学科和领域也能高屋建瓴。为此,我毅然来到他在华东师范大学的团队,拜师学艺。进入团队,我很快遇到了我的第三位引路人金澈清教授。金老师对待学术事无巨细且严谨有方,对待学生谆谆善诱。难以忘怀,在2017年国庆等节假日期间,金老师牺牲自己跟家庭团聚的时间帮我们修改论文。
在周、金两位老师的指导下,我不仅学术水平突飞猛进发表若干学术论文,而且项目管理和实现能力也得到较大提高,成功带领实验室同学完成两项校企合作项目。
除此之外,周、金两位老师在我母亲生病期间给我很大安慰和温暖,帮我度过了人生中最艰难的时刻。


其次,我十分感谢毛嘉莉老师,毛老师在我读博四年时光里,始终像知心大姐一样给我关心和鼓励。此外,在我几次投稿期间花费许多精力和时间与我讨论研究问题和研究方法,修改学术论文,让我学习了大量的学术论文写作技巧。
然后,我还要感谢南京师范大学赵斌老师。感谢赵老师始兄长般的热情和帮助。也特别感谢华东师范大学钱卫宁教授、王晓玲教授、何晓丰教授、张蓉教授、高明副教授和张召副教授在日常学习中与研究生课程中给予的多方面帮助和指导。


另外,感谢所有读书期间陪伴我的同学和朋友,你们是我的美好记忆。感谢已经毕业的孔超、房俊华和王科强等师兄,以及马建松、康强强、段小艺、孔扬鑫、宋秋革、包婷、杨小林和廖春和等师弟师妹,感谢你们在我研究生前期给予的学习上的帮助和生活上的快乐,难忘一起写论文和一起做项目的点点滴滴。
感谢王艺霖、杨小林、戚晓冬,你们为我的论文提出了宝贵的修改意见。
特别感谢陪我度过研究生时光的方祝和和肖冰,谢谢你们几年来对我的关心和包容。
感谢一起毕业的毛嘉莉博士、郭敬伟博士、刘辉平博士和朱涛博士,与你们一起毕业是我的荣幸。也祝尙在奋斗的乔宝强、戚晓冬博士、申弋斌博士、方祝和等博士顺利完成学业,早日毕业。感谢在LBS课题组一起学习的陈鹤、王婧、李敏茜、施晋、华嘉逊、濮敏等师弟师妹们。此外,我还想感谢燕存、李文俊、丁敬恩、徐寅等朋友,谢谢你们这些年一直对我的关心和照顾。

最后,着重感谢我的家人,对我无私的爱以及对我攻读博士学位的大力支持。

\end{spacing}
\vspace{0cm} \hspace{10.8cm}  章志刚

\hspace{9.8cm}  二零一八年四月 }

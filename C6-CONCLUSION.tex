\chapter{总结与展望}

本文重点研究了分布式轨迹数据上的$k$近邻查询。本章对全文研究内容和技术模型进行总结归纳,并展望未来的重点研究工作。

\section{研究总结}

随着移动设备的广泛应用,各行各业产生了海量轨迹大数据,随之而来面临着如何管理和利用该数据的问题。首先,本文研究了如何对采集到的轨迹数据进行有效管理并提供低延时的查询问题。本文的思路是利用现有分布式集群处理系统,构建基于分布式内存的轨迹管理系统。接着,本文针对海量轨迹数据散布在各个数据节点且无法被采集到集群内的场景,研究了$k$近邻查询的实现。本文的研究目标是在保证结果正确性的前提下,如何降低该查询的通信和时间开销。此外,目前存在着许多轨迹间距离度量方式。现有的查询研究工作往往只是针对某种距离进行查询优化,从而缺乏通用性。所以本文的另一个目标就是提出通用的$k$近邻查询处理方法。
%本文希望所提出的算法
%轨迹$k$近邻查询是轨迹数据挖掘研究的重要内容,也是基于位置服务的基本功能。
%针对海量轨迹数据,分布式场景下的$k$近邻查询问题应运而生。本文重点研究了基于协调者结点-远程结点分布式架构下的$k$近邻查询,
为此,本文从以下三个方面解决相关问题:
(1)\textit{如何对采集到的轨迹大数据进行有效管理并提供低延时的轨迹查询服务;}现有的分布式轨迹管理系统尽管能提供轨迹管理和查询服务,但由于这些系统都是基于磁盘的,导致查询执行期间I/O开销较大,无法提供低延时的查询效果。
(2)\textit{尽管目前存在着许多轨迹距离度量方式,但缺乏通用的查询处理策略}。设计统一的处理框架可以满足不同应用场景的需求,使得处理的问题不再受限。
(3)\textit{针对轨迹数据分布在各个数据节点上的$k$近邻查询,面临着通信和计算开销较大的问题;}相对于传统集中式环境下的查询,分布式环境下通信开销成为算法的首要瓶颈。此外,在降低通信开销的同时,仍需要保证查询效率。	
具体地,本文的研究问题和技术贡献总结如下。

首先,针对\textit{现有系统无法对海量轨迹数据提供实时查询服务问题},本文设计了基于分布式内存的管理系统TrajSpark。TrajSpark首先提出了基于分布式内存的轨迹数据表示结构IndexTRDD。该结构在引入了列存储的思想来表示轨迹片段并引入压缩技术以减少存储开销。此外,我们为IndexTRDD引入了全局和局部相结合的两层索引策略以降低查询的搜索开销。进一步的,为满足数据不断添加需求,TrajSpark引入了时间衰减模型以描述轨迹数据分布的变化,并使得系统能够自适应的动态划分新到来的数据。最后,TrajSpark针对三种典型轨迹查询设计了查询算法。海量轨迹数据集上的查询结果表明,TrajSpark相比已有分布式内存时空系统具有较高的优越性。

其次,针对\textit{现有轨迹距离度量方法多,导致 缺乏通用$k$近邻查询处理方法的问题},本文设计了两种基于多粒度概要数据的剪枝查询策略:一种是同时使用距离上、下界的剪枝策略,另一种是仅使用下界剪枝的策略。其中第一种策略适用于那些能够根据概要数据同时计算出上、下界的轨迹距离函数,第二种适用于那些仅能根据概要数据计算出下界的距离函数。针对具体的距离度量方式,我们只需研究出适合的概要数据并提供界信息,便能应用到对应的框架中。最后,研究了通过欧氏和动态时间弯曲两种距离在这两种策略下的应用,验证了策略的有效性。

最后,针对\textit{分布式$k$近邻查询中通信和计算开销过大的问题},本文设计的两种使用多粒度概要数据进行剪枝的策略能大大降低通信开销。
由于概要数据占用空间较少,通过发送概要数据来剪枝能够避免将数据发送到不必要的节点上。此外,随着数据粒度的增加,我们能够得到不断变紧的距离范围,从而进一步剪枝候选。在这两个策略中应用到欧氏和动态时间弯曲距离的过程,由于计算这两个距离的范围的开销远小于真实距离的计算开销,因而能同时达到降低计算开销的目的。

%本文基于多粒度概要数据模型和概要数据的界特征设计了两种查询处理框架:FTB和FLB。传统的近邻查询针对具体的距离度量方式,设计具体的查询实现算法。而本文提出的框架能满足不同距离度量方式的需求。其中FTB适用于那些根据概要数据能同时计算出距离上、下界的场景。而FLB适用于那些根据概要数据仅能计算出距离下界的场景。针对具体的距离度量方式,我们只需研究出适合的概要数据并提供界信息,便能应用到对应的框架中。

%其次,针对\textit{分布式查询的通信开销问题},本文强调了通信开销对查询方法性能的影响。现有查询研究着重于提高算法的时效性,鲜有考虑通信开销对算法性能的影响。在协调者-远程结点框架下,通信开销才是检验分布式算法的性能的首要指标。基于这一考量,本文设计的使用概要数据进行剪枝的思想能大大减少所要传输的数据量。进一步地,我们分别针对欧式距离设计了基于小波变换的多粒度概要数据,
%针对动态时间弯曲距离设计了基于包围信封的多粒度概要数据。本文设计的迭代式算法将概要数据由粗到细粒度发送到远程结点,并在每个结点进行剪枝。理论分析和基于公开数据集的实验表明本文所提方法能显著降低查询的通信开销。
	
%最后,针对\textit{分布式$k$近邻轨迹查询的时效问题},本文还考虑了如何在降低通信开销的同时,提高查询算法的时间性能。首先,本文提出的通过计算复杂度较低的距离上、下界进行剪枝的策略能有效地避免对所有候选轨迹进行复杂度较高的真实距离计算。
%此外,针对欧式距离应用场景,设计了在更新上、下界的过程中进行剪枝的策略以进一步降低计算开销。针对计算复杂度更高的动态时间弯曲距离应用场景,同时引入了使用索引树进行剪枝和在更新下界过程中剪枝的两种剪枝策略,以提高查询效率。理论分析和基于公开数据集的实验表明本文所提方法都具有较好时效性。

\section{研究展望}
本文重点研究了分布式$k$近邻轨迹查询问题。作者认为还可以从下面三个方面进行扩展。

首先,可以将本文系统进行更多的查询和应用扩展。本文所提出的系统TrajSpark目前支持基本的基于移动对象标识的查询和时空范围查询,以及高级的$k$近邻查询。在未来的工作中,我们将在TrajSpark上实现更多的查询及应用。

其次,可以针对语义轨迹上的查询进行拓展研究。本文所研究的轨迹数据来自欧氏空间,如何结合含非欧空间的语义信息进行近邻查询会是一个有趣的问题。一些前人工作\cite{Xiao,Kaiser,WangBCSSQ17}已经考虑了对语义轨迹上的查询。因此,基于轨迹的查询也是未来研究的重点。
	
%其次,本文研究了所提两个查询框架分别在欧式和动态时间弯曲这两个常用距离的具体实现。附件二还指出了在最长公共子串距离下的应用。但轨迹距离度量方式还有许多,在以后的工作中需要针对具体的度量方式设计适合的概要数据,并为概要数据提供距离范围计算方法。

最后,本文所提算法是针对协调者-远程结点这一分布式场景设计的查询算法。随着区块链\footnote{https://blockchain.info/}、以太坊\footnote{https://www.ethereum.org/}等基于P2P系统和概念的兴起,基于P2P这一分布式架构的轨迹等时间序列上的近邻查询会成为未来的热点。
\clearpage
\phantom{s}
\clearpage
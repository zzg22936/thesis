{\kaishu
\chapter*{致\qquad 谢}
\begin{spacing}{1.5}

四年本科、五年硕博,回首这九年的华东师大求学时光,历经本科好友、硕士同门的陆续毕业,此刻我也将博士毕业,感慨良多。在此期间,有幸得到老师和亲友们的指导与帮助,在此谨对他们表示衷心的感谢!
	
首先,我要郑重地感谢我的本硕博导师王晓玲教授。每个人生命中都会遇到贵人,我想她就是我的贵人。
她为我们创造了良好的学习环境与氛围,在学习方法上的细心指导,在生活上的关怀支持。我依稀记得刚王老师在我刚读研时为我指明了研究的方向,读博迷茫期时对我的耐心开导和鼓励,以及即将毕业之际传授于我未来需要的宝贵的工作经验。在以后的人生道路上,我都会一直铭记她的教诲。

其次,我十分感谢复旦大学的沙朝锋副教授。沙老师深厚的数学功底和精彩地模型介绍,让我看到了数据挖掘技术的魅力。感谢华东师范大学金澈清教授在关于LBS相关研究中给予的大量帮助和指导,金老师对待问题的严谨性让我印象深刻。然后,我还要感谢中国人民大学的赵鑫老师。赵老师花费许多精力和时间与我讨论研究问题和研究方法,修改学术论文,让我学习了大量的学术论文写作技巧。也特别感谢华东师范大学周傲英副校长、钱卫宁教授、宫学庆教授、何晓丰研究员、张蓉教授、高明副教授和周敏奇副教授在日常学习中与研究生课程中给予的多方面帮助和指导。

另外,感谢所有读研期间陪伴我的同学和朋友,你们是我的美好记忆。感谢已经毕业的林煜明博士、王立博士、徐辰博士、王朝勇、胡颢继等师兄,以及马建松、江俊文、段小艺等师弟师妹,感谢你们在我研究生前期给予的学习上的帮助和生活上的快乐。感谢张凯、彭宏伟、靳远远,你们为我的论文提出了宝贵的修改意见。特别感谢陪我度过研究生时光的朱涛和张新洲,谢谢你们几年来对我的关心和包容。感谢一起毕业的纪文迪博士、房俊华博士、孔超博士、张俍博士和孟丹博士,与你们一起毕业是我的荣幸。也祝尙在奋斗的朱涛博士、庞艳霞博士、毛嘉莉博士、章志刚博士、周欢博士顺利完成学业,早日毕业。感谢在109实验室一起学习的梁磊、赵大鹏、刘志、宋光旋、李财政、夏得伦、张颖、吕晓强、刘小捷、屈稳稳、贺韵宇、周纯依、刘文焱等师弟师妹们。此外,我还想感谢本科室友邱星星、吴超凡、李博,以及硕士同学张磊、李勇峰、董绍婵和顾玲,谢谢你们当年的一起玩耍以及对我找工作时的帮助和关心。

最后,着重感谢我的父母,对我攻读博士学位的大力支持。感谢未婚妻的一直陪伴,你的理解和奉献使我能够无忧地学习,感谢这份许多年来历久弥坚的爱恋。

\end{spacing}
\vspace{0cm} \hspace{10.8cm}  王科强

\hspace{9.8cm}  二零一七年五月 }

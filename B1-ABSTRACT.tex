%\vspace{-2.5cm}
\chapter*{\zihao{2}\heiti{摘~~~~要}}
%\vspace{-1cm}
近年来,随着手机等移动定位设备的大量使用,人们在生产生活中获得了飞速增长的轨迹大数据。该类大数据包含了车、人、动物甚至商品的移动行为。由于数据采集来源广、数据量大、数据增长快等原因,轨迹大数据往往以分布式形式存储。海量的轨迹数据不仅刻画了移动对象个体和群体的时空动态性,还蕴含着移动对象的行为信息,对交通导航、城市规划、物流调度等应用具有重要的价值。为充分挖掘轨迹数据的价值,学术界和工业界对轨迹数分析问题开展了大量研究工作,包括轨迹数据索引、轨迹聚类和分类、轨迹异常检测和移动预测等问题。

本文针对海量分布式轨迹数据,研究了$k$近邻查询,即从分布在不同数据结点的轨迹数据中找出与查询目标距离最近的$k$条轨迹。该查询广泛存在于基于位置服务的应用中,同时也是许多轨迹挖掘问题的子任务。
尽管这一查询可以通过直接将待查询轨迹数据发送到所有远程结点,并在其上进行两两距离计算以找出结果集。但该算法的通信复杂度过高,达到$O(n*M)$,其中$n$表示带查询轨迹的长度,$M$表示远程结点的数量。由于有限的网络带宽资源,无法实际应用。此外,由于需要查询轨迹跟所有轨迹进行距离值计算,导致其时间复杂度较高。
因此,该查询的重要挑战就是如何降低分布式算法的通信开销的同时快速给出查询结果。
为此,本文提出通过使用概要数据进行距离值估算并利用估计值进行剪枝的策略,从而达到节约通信开销的目的。
此外,现有轨迹距离度量方式较多,而现有工作往往只针对某一种方式进行处理。因此,缺乏能适用于所有距离准则的查询处理方案。
根据以上需求,设计了两种查询框架以满足各种距离度量方式的要求。第一种针对能根据概要数据同时估计出距离上、下界的场景。第二种针对那些只能根据概要数据估计出距离下界的场景。
接着,我们介绍了如何将具体的距离度量准则应用到这两种框架中。本文主要贡献包括如下三方面:

\begin{itemize}
	\item[1.] \textbf{设计了两种查询框架FTB和FLB以处理分布式$k$近邻轨迹查询。}
	在这两个框架中,我们将查询轨迹的多粒度概要数据,由粗到细发送到远程结点中。远程结点根据获取的概要数据进行距离范围估计,并利用估计值进行剪枝。随着所获概要数据粒度的增加,所估计的距离范围也越来越紧,剪枝后所剩候选也越来越少,直到只剩下$k$个候选。由于概要数据的大小远小于查询轨迹本身,因此使用概要数据进行剪枝的方式能显著减少通信开销。此外,使用概要数据计算估计值的时间也远小于计算真实距离的时间。从而使得这两个框架的效率也高。最后,这两个框架的主要区别在于FTB框架要求使用概要数据能同时估计出距离的上界和下界。而FLB框架仅要求能估计出下界。任一距离度量方式只要设计出能够满足距离估计的概要数据,就可以应用到对应框架中。
	
	\item[2.] \textbf{设计了ED-FTB算法以处理基于欧式距离的分布式$k$近邻轨迹查询。}欧式距离因其简单易算等特性,被广泛应用于时间序列数据分析。为此,本文研究了基于该距离准则的查询。我们首先使用哈尔小波对轨迹数据进行变换,得到多粒度的哈尔小波系数并使用其作为概要数据。接着,设计了基于部分哈尔小波系数的欧式距离上界和下界,并证明了随着系数数据的增加所得到的距离上、下界同时会越来越紧。基于以上结论,设计了欧式距离和FTB框架相结合的ED-FTB算法。在该算法中,引入了在计算上、下界过程中进行剪枝的策略以提高执行效率。最后,理论分析和实验结果相结合,展示了ED-FTB算法相比于基准算法的优越性。
	
	\item[3.] \textbf{设计了DTW-FLB算法以处理基于动态时间卷曲距离的分布式$k$近邻轨迹查询。}动态时间卷曲距离因允许两轨迹变化的速度不一样同样也被广泛应用时间序列数据的分析中。比如,在寻找相似的运动轨迹时,只要路径一样,哪怕速度或步行时间不一致也被认为是极相似的轨迹。为此,本文研究了基于动态时间卷曲距离的查询。首先,设计了满足动态时间卷曲约束的包围信封。在此基础之上,设计了多粒度包围信封。接着,设计了基于包围信封的动态时间卷曲距离下界,并证明了随着包围信封粒度的增加所得到的距离的下界会越来越紧。基于以上分析,设计了动态时间卷曲距离和FLB框架箱结合的DTW-FLB算法。同样在该算法中引入了在计算下界过程中进行剪枝和使用多个下界进行级联剪枝的策略以提高执行效率。最后,我们在真实轨迹数据集上验证了算法的有效性和可扩展性。
\end{itemize}
\hspace{-0.5cm}
\sihao{\heiti{关键词:}} \xiaosi{轨迹大数据,分布式$k$近邻查询,通信开销,时间效率,距离上、下界.}
%额外空白页